\documentclass{article}
\usepackage{amsmath}
\usepackage{amsfonts}
\usepackage{amssymb}
\renewcommand{\familydefault}{\sfdefault}

\title{String matching}

\begin{document}
\maketitle

\section{The scan left function}

\subsection{scan left}

\verb|scanl| is a polymorphic function of type \((\beta \to \alpha \to
\beta) \to \beta \to [\alpha] \to \beta\) for type variables,
\(\alpha\) and \(\beta\). It is defined by the following equations.
\begin{align*}
scanl\;f\;z\;[] &= [z] \\
scanlf\;z\;(x : xs) &= z : scanl\;f\;(f\;z\;x)\;xs
\end{align*}
%%\begin{verbatim}
%% let rec scanl (f : 'b -> 'a -> 'b) 
%%   (z : 'b) (xs : 'a list) : 'b list =
%%   z :: match xs with
%%        | [] -> []
%%        | (x :: xs) -> scanl f (f z x) xs
%% \end{verbatim}
%%\end{verbatim}
In C++, lists are replaced with iterators.
\begin{verbatim}
template <
  class F, class AccT, class InT, class OutT>
OutT scan_left (F f, AccT z, InT begin, InT end, OutT out) {
  *out++ = z;
  if (begin == end) return out;
  auto const& x = *begin;

  return scan_left (f, f (z, x), std::next (begin), end, out);
}
\end{verbatim}

\section{The string matching problem}

\subsection{matches}
A string matching problem is one in which one finds all occurrences of
a non-empty string (the pattern) in some other string (the text). A
specification for the problem can be stated like this:

\begin{equation}\label{equation:specification}
matches\;ws =map\;len \cdot filter\;(endswith\;ws) \cdot inits
\end{equation}

The function \(inits\) returns a list of the prefixes of the text in
order of increasing length. The expression \(endswith\;ws\;xs\) tests
whether the pattern \(ws\) is a suffix of \(xs\). Finally, the value
\(matches\;ws\;xs\) is a list of integers \(p\) such that \(ws\) is a
suffix of \(take\; p\;xs\).  For example:
\(matches\;"abcab"\;"ababcabcab"\) is the list \([7, 10]\).  That is,
\(matches\;ws\;xs\) returns a list of integers \(p\) such that \(ws\)
appears in \(xs\) ending at position \(p\) (counting positions from
\(1\)).

\subsection{filter}
In C++, \(filter\) can be written like this.
\begin{verbatim}
template <class PredT, class RngT, class OutT>
OutT filter (PredT p, RngT xs, OutT out) {
  return std::accumulate (
      std::begin (xs), std::end (xs), out, 
      [&p](auto dst, auto const& x) { 
           return p (x) ? *dst++ = x : dst;
      }
  );
}
\end{verbatim}
%% \subsection{inits}
%% \verb|inits| itself can be implemented in terms of \verb|scanl|
%% \begin{align*}
%% inits = scanl\;snoc\;[\;]
%% \end{align*}
%% where
%% \begin{align*}
%% snoc\;x\;a = x\;@\;[a]
%% \end{align*}
%% \begin{verbatim}
%% let inits (xs : char list) : char list list = 
%%   scanl (fun x a -> x @ [a]) [] xs
%% \end{verbatim}
%% This alternative definition shows early motivation for the scan lemma.
%% \begin{verbatim}
%% let inits (xs : char list) : char list list = 
%%   let f (h :: tl) c = (c :: h) :: h :: tl in
%%   List.rev @@ List.map (List.rev) (List.fold_left f [[]] xs)
%% \end{verbatim}

\subsection{endswith}
We define \(endswith\;ws = (reverse\;ws\sqsubseteq)\cdot reverse\)
where \( \sqsubseteq \) is the \(prefix\) relation given by the
equations
\begin{gather*}
[\;]\;\sqsubseteq us = true \\
(u:us) \sqsubseteq [\;] = false \\
(u:us) \sqsubseteq (v:vs) = (u = v \land us \sqsubseteq vs)
\end{gather*}
Here's the \(prefix\) function realized in C++.
\begin{verbatim}
template <class InT1, class InT2>
bool prefix (InT1 lb, InT1 le, InT2 rb, InT2 re) {
  if (lb == le) return true;
  if (rb == re) return false;

  return *lb == *rb && 
         prefix (std::next (lb), le, std::next (rb), re);
}
\end{verbatim}
%% \begin{align*}
%%   endswith\;ws\;xs  = reverse\;ws \sqsubseteq reverse\; xs
%% \end{align*}

\section{The Boyer-Moore solution}
\subsection{Theory}
This identity is called ``the scan lemma''.
\begin{equation}
map\;(foldl\;op\;e) \cdot inits = scanl\;op\;e
\end{equation}
This equation is important because the left hand side has complexity
\(O(N^{2})\) whereas the right-hand-side, \(O(N)\). It admits
restating equation \ref{equation:specification} as:
\begin{gather}
matches\;ws = map\;fst \cdot filter ((sw \sqsubseteq) \cdot snd)\;scanl\; step\;(0, []) \nonumber\\
sw = reverse\;ws \nonumber\\
step\;(n, sx)\;x = (n + 1, x:sx)
\end{gather}
This is the called the basic ``Boyer-Moore'' algorithm. That's what
we'll implement in C++.
\begin{verbatim}
template <class OutT>
OutT matches (std::string const& ws, std::string const& s, OutT dst) {
  typedef std::pair<int, std::deque<char>> acc_t;

  auto step = [](acc_t p, char x) -> acc_t {
    ++p.first;
    p.second.push_front (x);

    return p;
  };

  std::deque<acc_t> buf;
  scan_left (
      step
    , std::make_pair(0, std::deque<char>())
    , s.begin ()
    , s.end ()
    , std::back_inserter(buf));

  std::string sw(ws.rbegin (), ws.rend ());
  auto pred = [&sw] (auto p) -> bool { 
    return prefix (
      sw.begin (), sw.end ()
    , p.second.begin (), p.second.end ()); 
  };
  std::deque<acc_t> temp;
  filter (pred, buf, std::back_inserter (temp));

  return std::transform (
     temp.begin (), temp.end (), dst, 
     [](acc_t const& p) -> int {  return p.first; });
}
\end{verbatim}
\section{Testing}
Here's a little test driver.
\begin{verbatim}
int main () {

  std::list<int> where;
  matches ("abcab", "ababcabcab", std::back_inserter (where));

  std::for_each (where.begin (), where.end ()
   , [](int i) -> void { std::cout << i << ", "; }
   );  

  return 0;
}
\end{verbatim}
This program should print ``7, 10''.
\end{document}
