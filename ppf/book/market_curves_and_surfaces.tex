\chapter{Market -- Curves and
Surfaces}\label{ch:market-curves-surfaces}

A financial market is made up out of lots of pieces of information. For example, 
foreign exchange rates, equity spot prices, yield curves, volatility surfaces
and correlation surfaces, to name a few. In this chapter we discuss simple classes 
for representing curves and surfaces provided in the \verb|ppf.market|
sub-package. The chapter is concluded with a discussion of the market
environment which is designed to represent a financial market.


\section{Curves}
Fundamentally, without regard to the specific market variable being modelled
(e.g. discount factors, forward rates, volatilities at a fixed strike
and expiry for a range of tenors), a curve is the association between a set
of points at which the function is known (abscissae) and the known
function values at those points (ordinates) and an interpolation
algorithm for estimating the value of the function between the known
abscissae. Utilisation of the interpolation algorithms presented in
section \ref{sec:interpolation} deals with the `heavy lifting' of
curve representation. From the \verb|ppf.market.curve| module we have
the simple curve
\begin{verbatim}
class curve:
  def __init__(self, times, factors, interp):
    self.__impl = interp(times, factors)
  def __call__(self, t): return self.__impl(t)
\end{verbatim}
The following interpreter transcript (simulating a discount factor
curve) shows how easily a curve can be modelled by these components:
\begin{verbatim}
  >>> import math
  >>> from ppf.math.interpolation import loglinear
  >>> times = range(0, 22)
  >>> factors = [math.exp(-0.05*T) for T in times]
  >>> P = curve(times, factors, loglinear)
  >>> for t in times: print P(t)
  1.0
  0.951229424501
  0.904837418036
  0.860707976425
  0.818730753078
  0.778800783071
  0.740818220682
  0.704688089719
  0.670320046036
  0.637628151622
  0.606530659713
  0.57694981038
  0.548811636094
  0.522045776761
  0.496585303791
  0.472366552741
  0.449328964117
  0.427414931949
  0.406569659741
  0.386741023455
  0.367879441171
  0.349937749111
\end{verbatim}

\section{Surfaces}
A surface is the common name ascribed to a model of a multi-variabled
function. In the interest of brevity, we restrict our attention to
modelling functions in just two variables $f = f(x, y)$. The
\verb|ppf.market.surface| module offers the \verb|class surface| for
their representation which is associated with the commonly encountered
bilinear-interpolation scheme that relies on the
\verb|ppf.utility.bound| function (refer to section
\ref{sec:interpolation}):
\begin{verbatim}
import ppf.utility

class surface:
  def __init__(self, first_axis, second_axis, values):
    self.__first_axis = first_axis
    self.__second_axis = second_axis
    self.__values = values

  def __call__(self, x, y):
    i1, i2 = ppf.utility.bound(x, self.__first_axis)
    j1, j2 = ppf.utility.bound(y, self.__second_axis)
    f = self.__values
    x1 = self.__first_axis[i1]
    x2 = self.__first_axis[i2]
    y1 = self.__second_axis[j1]
    y2 = self.__second_axis[j2]
    r = (x2 - x1)*(y2 - y1)
    return (f[i1, j1]/r)*(x2 - x)*(y2 - y) + \
           (f[i2, j1]/r)*(x - x1)*(y2 - y) + \
           (f[i1, j2]/r)*(x2 - x)*(y - y1) + \
           (f[i2, j2]/r)*(x - x1)*(y - y1)
\end{verbatim}
The \verb|ppf.test.test_market| module provides us with an example of
using instances of this type for surface representation, in this case
an expiry-tenor volatility surface:
\begin{verbatim}
class surface_tests(unittest.TestCase):
  def test(self):
    from ppf.date_time import date
    from ppf.date_time import months
    from ppf.date_time import Feb, Apr, Jul, Oct, Jan
    from numpy import zeros

    expiries = [
       date(2006, Feb, 11)
     , date(2006, Apr, 11)
     , date(2006, Jul, 11)
     , date(2006, Oct, 11)
     , date(2007, Jan, 11)
     , date(2008, Jan, 11)
     , date(2009, Jan, 11)
     , date(2010, Jan, 11)
     , date(2011, Jan, 11)
     , date(2012, Jan, 11)
     , date(2013, Jan, 11) ]

    tenors = [ months(12), months(24), months(36) ]

    vols = zeros((len(expiries), len(tenors)))
    # expiry, tenor surface                    1y    2y     3y                                
    vols[ 0, 0], vols[ 0, 1], vols[ 0, 2] = 200.00, 76.25, 64.00 # 1m
    vols[ 1, 0], vols[ 1, 1], vols[ 1, 2] =  98.50, 84.75, 69.00 # 3m
    vols[ 2, 0], vols[ 2, 1], vols[ 2, 2] =  98.00, 81.75, 68.00 # 6m
    vols[ 3, 0], vols[ 3, 1], vols[ 3, 2] = 101.25, 82.25, 69.25 # 9m
    vols[ 4, 0], vols[ 4, 1], vols[ 4, 2] = 106.00, 82.00, 69.25 # 1y
    vols[ 5, 0], vols[ 5, 1], vols[ 5, 2] =  78.75, 73.25, 61.25 # 2y
    vols[ 6, 0], vols[ 6, 1], vols[ 6, 2] =  66.25, 59.00, 50.00 # 3y
    vols[ 7, 0], vols[ 7, 1], vols[ 7, 2] =  55.25, 47.75, 41.75 # 4y
    vols[ 8, 0], vols[ 8, 1], vols[ 8, 2] =  44.75, 40.25, 35.50 # 5y
    vols[ 9, 0], vols[ 9, 1], vols[ 9, 2] =  32.00, 30.50, 28.25 # 6y
    vols[10, 0], vols[10, 1], vols[10, 2] =  26.50, 24.25, 24.25 # 7y

    base = date(2006, Jan, 11);
    sig = ppf.market.surface(
         [int(t - base)/365.0 for t in expiries]
       , [m.number_of_months().as_number() for m in tenors]
       , vols)

    tol=1.0e-8
    for i in range(len(expiries)):
      expiry = expiries[i]
      t = int(expiry - base)/365.0
      for j in range(len(tenors)):
        tenor = tenors[j]
        T = tenor.number_of_months().as_number()
        assert math.fabs(sig(t, T) - vols[i, j]) <= tol 
\end{verbatim}

\section{Environment} \label{sec:market}

For the purposes of pricing it is often convenient to aggregrate all
the different pieces of market data (e.g. curves and surfaces) into a
single container. The class \verb|environment| from
\verb|ppf.market.environment| provides us with a simple container for
all the bits of market data. In addition the class is also the single
point of access to the market data for the pricing models. As well as
containing surfaces and curves, the environment also contains
constants. An example of a constant could be the speed of mean
reversion used in the Hull White model to control terminal
correlation.

An environment is constructed with a pricing date and pieces of market data are 
added to it using the \verb|add| methods. Each piece of data is stored against a unique 
key which is later used to retrieve the data via the \verb|retrieve| methods. Typically in 
a fully productionized pricing framework, there would be another level of indirection 
to insulate clients of the libraries from the internal representation of the market data keys. 
Or more specifically, one would write environment factories taking in a dictionary of market data 
against the physical names and map the physical names to the market data keys.   
\begin{verbatim}
import ppf.date_time

class environment:
  def __init__(self, pd = ppf.date_time.date(2008, 01, 01)):
    self.pd = pd
    self.curves = {}
    self.surfaces = {}
    self.constants = {}

  def pricing_date(self):
    return self.pd

  def relative_date(self, d):
    ret = ppf.date_time.days.days(d-self.pd)
    if ret < 0:
      raise RuntimeError, 'date before pricing date'
    return ret
    
  def add_curve(self, key, curve):
    if self.curves.has_key(key):
      del self.curves[key]
    self.curves[key] = curve 

  def add_surface(self, key, surface):
    if self.surfaces.has_key(key):
      del self.surfaces[key]
    self.surfaces[key] = surface

  def add_constant(self, key, constant):
    if self.constants.has_key(key):
      del self.constants[key]
    self.constants[key] = constant 

  def has_curve(self, key):
    return self.curves.has_key(key)

  def has_surface(self, key):
    return self.surfaces.has_key(key)

  def has_constant(self, key):
    return self.constants.has_key(key)

  def retrieve_curve(self, key):
    if not self.has_curve(key):
      raise RuntimeError, 'unable to find curve'
    return self.curves[key]

  def retrieve_surface(self, key):
    if not self.has_surface(key):
      raise RuntimeError, 'unable to find surface'
    return self.surfaces[key]

  def retrieve_constant(self, key):
    if not self.has_constant(key):
      raise RuntimeError, 'unable to find constant'
    return self.constants[key]
\end{verbatim}
Later chapters will make use of instances of \verb|class environment|
constantly. An example from the \verb|ppf.test.test_hull_white| module
should suffice to give an idea of its usage for now:
\begin{verbatim}
class fill_tests(unittest.TestCase):
  def test_numeraire_rebased_bond(self):
    env = ppf.market.environment()
    times = numpy.linspace(0, 2, 5)
    factors = numpy.array([math.exp(-0.05*t) for t in times])
    env.add_curve("zc.disc.eur"
        , ppf.market.curve(times, factors, ppf.math.interpolation.loglinear))
    expiries, tenors =      [0.1, 0.5, 1.0, 1.5, 2.0, 3.0, 4.0, 5.0], [0, 90]
    env.add_surface("ve.term.eur.hw"
                 , ppf.market.surface(expiries, tenors, numpy.zeros((8, 2))))
    env.add_constant("cv.mr.eur.hw", 0.0)
    r = ppf.model.hull_white.requestor()
    s = ppf.model.hull_white.lattice.state(11, 3.5)
    sx = s.fill("eur", 0.25, r, env)
    f = ppf.model.hull_white.fill(2.0)
    PtT = f.numeraire_rebased_bond(0.25, 1.5, "eur", env, r, sx)
    exp = \
        [1.02531512052
        ,1.02531512052
        ,1.02531512052
        ,1.02531512052
        ,1.02531512052
        ,1.02531512052
        ,1.02531512052
        ,1.02531512052
        ,1.02531512052
        ,1.02531512052
        ,1.02531512052]

    _assert_seq_close(exp, PtT)
\end{verbatim}
