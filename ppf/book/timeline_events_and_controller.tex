\chapter{Timeline -- Events and
Controller}\label{ch:timeline-events-context}
In the preceding chapter we looked at trade representation. This
chapter covers a number of concepts that provide the glue between the
trade representation and the pricing models. We will first discuss the
concept of an event. Once we have a firm understanding of an event we
will move onto the timeline, which is essentially a convenient
container for a sequence of events. We finish the chapter by reviewing
the idea of a controller: of all the classes discussed in this
chapter, the controller is the conduit through which pricing models
communicate with the trade representation.
 
\section{Events}

A fundamental concept of any pricing analytics is that of an event. So
what does an event represent? In essence an event either represents a
cashflow depending on some piece of financial information, such as a
foreign exchange rate on a particular date for example, or an exercise
decision. The module \verb|ppf.core.event|, as outlined below,
contains two core classes: \verb|pay_event| and \verb|exercise_event|.
The class \verb|pay_event| is constructed from a flow, a pay or
receive flag, a leg id, and a reset id. The leg id enables clients to
determine which leg of the trade the flow belongs to and the reset id
informs clients which observable of the flow is represented by the
event. The class \verb|exercise_event| is constructed from an exercise
and a flag representing the type of exercise, i.e. cancellable or
callable. For completeness two helper functions are provided for
determining the event type. We will use these helper functions in the
forthcoming chapters.
   
\begin{verbatim}
import string
class event_type:
    flow, exercise = (1, -1)

class pay_event:
  def __init__(
        self
      , flow
      , pay_rcv
      , leg_id
      , reset_id):
    self.__flow    \
  , self.__pay_rcv \
  , self.__leg_id \
  , self.__reset_id = flow, pay_rcv, leg_id, reset_id

  def flow(self) : return self.__flow
  def pay_recieve(self) : return self.__pay_rcv
  def leg_id(self) : return self.__leg_id
  def reset_id(self) : return self.__reset_id
  def pay_currency(self) : return self.__flow.pay_currency()
  def __str__(self) :
    s = "payment [%s, %s, %s, %s], " % \
         (self.__pay_rcv, self.__leg_id, self.__reset_id, str(self.__flow))
    return s

class exercise_event:
  def __init__(
        self
      , exercise_opportunity
      , exercise_type):
    self.__exercise_opportunity  \
  , self.__exercise_type = exercise_opportunity, exercise_type
  def exercise_type(self) : return self.__exercise_type
  def exercise_opportunity(self) : return self.__exercise_opportunity
  def pay_currency(self) : return self.__exercise_opportunity.fee_currency()
  def __str__(self) :
    s = "exercise [%s, %s], " % \
        (self.__exercise_type, str(self.__exercise_opportunity))
    return s

def is_pay_event(event): return isinstance(event, pay_event)
def is_exercise_event(event): return isinstance(event, exercise_event)
\end{verbatim}

\section{Timeline}
Now that we have a clear understanding of an event, we move on to the
timeline. Any financial payoff can be interpreted as a sequence of
cash flows and exercise decisions with potentially more than one
cashflow occurring at any one point in time. The purpose of the class
\verb|timeline| from the \verb|ppf.core.timeline| module is to
transform the trade representation into a sequence of events. The
actual transformation occurs in the constructor of the timeline. We
begin by harvesting the observables from the trade to create instances
of the \verb|pay_event| and follow on by harvesting the exercises from
the trade to create instances of the \verb|exercise_event|. Once
constructed, an instance of the timeline can be queried for a list of
the times, on which either a cash flow occurs or an exercise decision
is made, and for a list of events occurring at a particular time.
\begin{verbatim}
from types import *
from trade import *
from leg import *
from flow import *
from exercise import *
from event import *

class timeline:
  def __add_event_(self, t, event):
    if not self.__events.has_key(t.julian_day()):
      self.__events[t.julian_day()] = []
    self.__events.get(t.julian_day()).append(event)

  def __init__(self, trade, pricing_date):
    self.__events = {}

    # add events from legs
    leg_id = 0
    for l in trade.legs():
      pay_rcv = l.pay_receive()
      for f in \
        [f for f in l.flows()
         if f.pay_date() >= pricing_date]:
        observables = f.observables()
        if not observables:
          raise RuntimeError, "Missing observables"
        for o in observables:
          self.__add_event_(
              o.reset_date()
            , pay_event(f, pay_rcv, leg_id, o.reset_id()))
      leg_id += 1

    # add events from exercise schedule
    if trade.has_exercise_schedule():
      ex_type = trade.exercise_type()
      for ex in \
        [ex for ex in trade.exercise_schedule()
         if ex.notification_date() > pricing_date]:
        self.__add_event_(
            ex.notification_date()
          , exercise_event(ex, ex_type))

  def times(self):
    return sorted(self.__events.keys())

  def events(self, t):
    return self.__events[t]

  def __str__(self):
    s = "events: \n"
    times = sorted(self.__events.keys())
    for t in times:
      s += "\"%s\", " % t
      events = self.__events[t]
      for event in events: s += str(event)
      s += '\n'
    return s
\end{verbatim}
Construction of a timeline from a financial structure proceeds as per
the following example:
\begin{verbatim}
  >>> from ppf.date_time import *
  >>> from pay_receive import *
  >>> from generate_flows import *
  >>> from generate_observables import *
  >>> from generate_exercise_table import *
  >>> from exercise_type import *
  >>> from leg import *
  >>> from trade import *
  >>> libor_observables = generate_libor_observables(
  ...     start  = date(2007, Jun, 29)
  ...   , end  = date(2009, Jun, 29)
  ...   , roll_period = 6
  ...   , roll_duration = ppf.date_time.months
  ...   , reset_period = 3
  ...   , reset_duration = ppf.date_time.months
  ...   , reset_currency = "JPY"
  ...   , reset_basis = basis_act_360
  ...   , reset_shift_method = shift_convention.modified_following)
  >>> coupon_observables = generate_fixed_coupon_observables(
  ...     start  = date(2007, Jun, 29)
  ...   , end  = date(2009, Jun, 29)
  ...   , roll_period = 6
  ...   , reset_currency = "JPY"
  ...   , coupon_shift_method = shift_convention.modified_following
  ...   , coupon_rate = 0.045)
  >>> #semi-annual flows
  >>> pay_flows = generate_flows(
  ...   start  = date(2007, Jun, 29)
  ...   , end  = date(2009, Jun, 29)
  ...   , duration = ppf.date_time.months
  ...   , period = 6
  ...   , shift_method = shift_convention.modified_following
  ...   , basis = "30/360"
  ...   , observables = coupon_observables)
  >>> rcv_flows = generate_flows(
  ...   start  = date(2007, Jun, 29)
  ...   , end  = date(2009, Jun, 29)
  ...   , duration = ppf.date_time.months
  ...   , period = 6
  ...   , shift_method = shift_convention.modified_following
  ...   , basis = "A/360"
  ...   , observables = libor_observables)
  >>> pay_leg = leg(pay_flows, PAY)
  >>> receive_leg = leg(rcv_flows, RECEIVE)
  >>> #1y nc
  >>> ex_sched = generate_exercise_table(
  ...   start = date(2008, Jun, 29)
  ... , end  = date(2009, Jun, 29)
  ... , period = 1
  ... , duration = ppf.date_time.years
  ... , shift_method = shift_convention.modified_following)
  >>> structure = trade((pay_leg, receive_leg), (ex_sched, exercise_type.callable))
  >>> pricing_date = date(2007, Jan, 29)
  >>> tline = timeline(structure, pricing_date)
\end{verbatim}

\section{Controller}
As already discussed in the introduction, the controller provides the
main glue between the trade representation (or more correctly its
transformation into events) and the pricing models. Essentially the
controller is constructed from the trade, the model and the market
environment. The final argument of the constructor is used to control
how historical discount factors are treated and we defer further
discussion on this point to later chapters. The main job of the
controller is to maintain a dictionary of variables whose values will
be updated and retrieved by the pricing models. The controller is also
the conduit for the pricing models to evaluate the payoffs in the
trade representation.  The payoffs are evaluated by first setting the
event followed by an invocation of the function call operator.  The
following excerpt from the \verb|ppf.core.controller| module
illustrates the essential details of the controller class. 
\begin{verbatim}
class controller:
  def __init__(self, trade, model, env, historic_df):
    self.__trade = trade
    self.__model = model
    self.__env = env
    self.__historical_df = historical_df
    self.__symbol_table = {}
    self.__event = None

  def get_trade(self):
    return self.__trade

  def get_model(self):
    return self.__model

  def get_environment(self):
    return self.__env

  def get_event(self):
    return self.__event

  def set_event(self, event):
    self.__event = event

  def insert_symbol(self, name, at):
    self.__symbol_table[name] = (at, self.__model.state().create_variable())

  def update_symbol(self, name, symbol, at):
    self.__symbol_table[name] = (at, symbol)

  def retrieve_symbol(self, name):
    if not self.__symbol_table.has_key(name):
      raise RuntimeError, "name not found in symbol table"
    return self.__symbol_table.get(name)[1]

  def retrieve_symbol_update_time(self, name):
    if not self.__symbol_table.has_key(name):
      raise RuntimeError, "name not found in symbol table"
    return self.__symbol_table.get(name)[0]
\end{verbatim}
In the body of the \verb|__call__| method we retrieve the leg corresponding to the
event from the trade, the payoff from the leg and determine whether
the leg is pay or receive. The payoff is then evaluated by calling the
function call operator on the payoff class, passing in the controller
as an argument.  
\begin{verbatim}
class controller:
  def __call__(self, t):
    leg = self.__trade.legs()[self.__event.leg_id()]
    payoff = leg.payoff()
    pay_rcv = leg.pay_receive()
    return pay_rcv*payoff(t, self)         
\end{verbatim}
In future chapters we will provide concrete examples
of payoff classes. For the time being all we insist upon is that
the payoff class provides an implementation of the function call
operator with the correct signature.

In later chapters, we will return to the controller to add more
methods. In contrast to strongly typed programming languages such as
C++, the addition of more methods on the controller doesn't create an
(compile/link dependency) implementation bottleneck because if a call
onto a \verb|controller| method results in invoking a call on a method
of a particular model that hasn't been implemented, all we get is a
runtime error. To achieve this kind of flexibility in C++ requires
more effort but can be reasonably dealt with interface based
programming paradigms such as one finds in COM \footnote{A footnote on
the introductory page of chapter \ref{ch:python-excel-integration}
explains the meaning of the acronym COM.}.

