\chapter {Python} \label{appendix:python-tutorial}

In this appendix, we provide a whirlwind tour of the Python
programming language. It will in no way be exhaustive and is not
intended to negate the need for learning the language more
comprehensively using the many excellent resources available teaching
Python\footnote{A good place to start is the "Python
tutorial" available online at http://docs.python.org/tutorial/}. 
Nonetheless, it should be enough for a newcomer to the language to get
started.

\section{Python interpreter modes}

The Python interpreter can be executed in one of two modes,
`interactive' or `batch'. When run interactively in a shell or on
Windows, at a command prompt, the interpreter will wait for input from
the user and when sufficient input has been made for a statement to be
executed will execute that statement immediately and go back to
waiting for the next. In batch mode, a complete Python script stored
in a file is given to the interpreter and executed all in one go.

\subsection{Interactive mode}

To launch an interactive session with the interpreter, from your shell
or command prompt, issue the command \verb|python -i|. All being well,
the interpreter should then indicate its willingness to begin
processing statements by displaying its \emph{primary} prompt which
looks like a sideways chevron `\verb|>>>|'. On Windows, starting a
session might look something like this:
\begin{verbatim}
c:\Documents and Settings\PythonUser>python -i
python -i
ActivePython 2.5.1.1 (ActiveState Software Inc.) based on
Python 2.5.1 (r251:54863, May  1 2007, 17:47:05) [MSC v.1310 32 bit (Intel)] on win32
Type "help", "copyright", "credits" or "license" for more information.
>>> 
\end{verbatim}
When the interpreter has seen some part of a statement, but not
sufficiently enough to execute anything, the Python interpreter will
indicate that situation by displaying its \emph{secondary} prompt
which looks like `\verb|...|'.

\subsection{Batch mode}

To have the interpreter execute a complete Python script stored in a
file on disk, simply run the Python executable passing the name of the
script to be executed. For example, assuming the existence of a file
`hello\_world.py' (files containing Python script by convention have
a `.py' suffix), we might execute it like so:
\begin{verbatim}
c:\Documents and Settings\PythonUser>python hello_world.py
\end{verbatim}

\section{Basic Python}

\subsection{Simple expressions}

The Python interpreter can be used like a calculator to evaluate
numerical expressions:
\begin{verbatim}
>>> 1 + 2 + 3
6
\end{verbatim}
Number valued expressions in Python are of integer or floating point
type:
\begin{verbatim}
>>> 22/7
3
>>> 22/7.0
3.1428571428571428
>>> 
\end{verbatim}
String literal expressions are values too:
\begin{verbatim}
>>> "Hello world!"
'Hello world!'
>>> "Goodbye cruel " + "world."
'Goodbye cruel world.'
>>> 
\end{verbatim}
In the last example, we made use of the string concatenation operator
`\verb|+|' to concatenate two string literal expressions into one.

The value of an expression can be associated with a variable using the
\emph{assignment operator} denoted `\verb|=|':
\begin{verbatim}
>>> x = 1 + 2 + 3
\end{verbatim}
Doing so means the evaluation of the named expression can be referred
to again later:
\begin{verbatim}
>>> print x - 6
0
\end{verbatim}
Assigning a new expression to an existing variable causes the
expression the variable was associated with to be discarded and the
variable to become associated with the new expression instead:
\begin{verbatim}
>>> bar = "baz"
>>> print bar
baz
>>> bar = 42
>>> print bar
42
>>> 
\end{verbatim}
Comments are indicated by a `\verb|#|' and are ignored by the
interpreter:
\begin{verbatim}
>>> #this is a comment!
... 
>>> 
\end{verbatim}

\subsection{Built-in data types}

We have seen that natively Python supports integers, floating point
values and character strings. Python also natively provides incredibly
useful heterogeneous container types. The first of these is the
\emph{tuple}. A tuple is a fixed collection of values and can be
constructed using parentheses like so:
\begin{verbatim}
>>> t = (1, 2, "foo")
>>> print t
(1, 2, 'foo')
>>> \end{verbatim}
Naturally, tuples can contain values of any type including tuples:
\begin{verbatim}
>>> s = (t, t, (t,))
>>> print s
((1, 2, 'foo'), (1, 2, 'foo'), ((1, 2, 'foo'),))
>>> 
\end{verbatim}
The value at the $i$-th position of a tuple can be retrieved like this:
\begin{verbatim}
>>> print t[0]
1
>>> print t[1]
2
>>> print t[2]
foo
>>> print t[3] #uh-oh...
Traceback (most recent call last):
  File "<stdin>", line 1, in <module>
IndexError: tuple index out of range
>>> 
\end{verbatim}
Tuples can be used to assign to multiple values very concisely:
\begin{verbatim}
>>> x,y,z=t
>>> print "%d,%d,%s" % (x, y, z)
1,2,foo
\end{verbatim}
Tuples are immutable which means a tuple value may not be modified:
\begin{verbatim}
>>> t[0]=10
Traceback (most recent call last):
  File "<stdin>", line 1, in <module>
TypeError: 'tuple' object does not support item assignment
>>> 
\end{verbatim}
In fact, integers, floats, strings and tuples are all immutable types.

When a mutable sequence is needed, Python steps in with its
\emph{list} type. Lists are created using square brackets like this:
\begin{verbatim}
>>> l = [1, 1.0, "one"]
>>> print l
[1, 1.0, 'one']
>>> 
\end{verbatim}
Lists contain values of any type (built-in or user-defined) including
of course, tuples and lists.
\begin{verbatim}
>>> m = [l, t]
>>> print m
[[1, 1.0, 'one'], (1, 2, 'foo')]
>>> 
\end{verbatim}
Lists as mentioned are a mutable type:
\begin{verbatim}
>>> l[0], l[1], l[2] = (2, 2.0, 'two')
>>> print l
[2, 2.0, 'two']
>>> 
\end{verbatim}
Be careful! Because lists are not immutable one needs to be mindful of
side-effects resulting from aliasing:
\begin{verbatim}
>>> ll = l
>>> ll[0] = 3
>>> print l
[3, 2.0, 'two']
>>> 
\end{verbatim}
Do you see what's happened there? The value referred to by \verb|l|
was modified through an alias \verb|ll|. 

Now, Python has very powerful constructs for creating lists termed
\emph{list comprehensions}:
\begin{verbatim}
>>> u = [i for i in range(10)]
>>> print u
[0, 1, 2, 3, 4, 5, 6, 7, 8, 9]
>>> v = [i*i for i in u]
>>> print v
[0, 1, 4, 9, 16, 25, 36, 49, 64, 81]
>>> print [x for x in v if x > 25]
[36, 49, 64, 81]
\end{verbatim}
List comprehensions are very much worth finding out about as early as
possible.

Data structures commonly known as \emph{associated arrays} in other
programming languages are termed \emph{dictionaries} in Python. They
are arrays accessed by \emph{keys} where each key is associated with a
value. An empty dictionary is denoted `\verb|{}|':
\begin{verbatim}
>>> d = {}
>>> print d
{}
\end{verbatim}
A non-empty dictionary can be created by passing a sequence of
key-value pairs like so:
\begin{verbatim}
>>> d = {"1":1, "2":2, "3":3}
>>> print d
{'1': 1, '3': 3, '2': 2}
\end{verbatim}
Naturally, dictionary values may be of any type including lists and
tuples:
\begin{verbatim}
>>> d["list"] = l
>>> d["tuple"] = t
>>> print d
{'1': 1, '3': 3, '2': 2, 'list': [1, 1.0, 'one'], 'tuple': (1, 2, 'foo')}
\end{verbatim}
Any immutable type will serve for a dictionary key including tuples
(as long as the values of the tuple are themselves all immutable):
\begin{verbatim}
>>> m = {(1, 2, 3):"a tuple"}
>>> print m
{(1, 2, 3): 'a tuple'}
>>> 
\end{verbatim}
As previously noted, lists are not immutable:
\begin{verbatim}
>>> m = {[1, 2, 3]:"a list"}
Traceback (most recent call last):
  File "<stdin>", line 1, in <module>
TypeError: list objects are unhashable
>>> 
\end{verbatim}

\subsection{Control flow statements}

Let's start with simple iteration using the Python \verb|for|
statement:
\begin{verbatim}
>>> for i in range(10):
...   print i
... 
0
1
2
3
4
5
6
7
8
9
>>> 
\end{verbatim}
Encountered earlier in this piece but not described is the function
\verb|range|. Maybe it's better to delegate explanation to the
doc-string for \verb|range| like this:
\begin{verbatim}
>>> help(range)
Help on built-in function range in module __builtin__:

range(...)
    range([start,] stop[, step]) -> list of integers
    
    Return a list containing an arithmetic progression of integers.
    range(i, j) returns [i, i+1, i+2, ..., j-1]; start (!) defaults to 0.
    When step is given, it specifies the increment (or decrement).
    For example, range(4) returns [0, 1, 2, 3].  The end point is omitted!
    These are exactly the valid indices for a list of 4 elements.

>>> 
\end{verbatim}
Now returning to our understanding of the \verb|for| statement example
above:
\begin{verbatim}
>>> for i in range(10):
...   print i
... 
\end{verbatim}
Note how the \verb|print i| is indented relative to the preceding line
(\verb|for i in ...|). This is significant. That is, white space is
significant in Python and its use indicates where different parts of
statements begin and end much like the use of `\verb|{|' and
`\verb|}|' in C/C++. To illustrate further, here's a more involved
snippet:
\begin{verbatim}
>>> for i in range(0, 2):
...   print i
...   for j in range(0, 10):
...     print "  %d" % (j)
...   print
... 
0
  0
  1
  2
  3
  4
  5
  6
  7
  8
  9

1
  0
  1
  2
  3
  4
  5
  6
  7
  8
  9
>>>
\end{verbatim}
Dropping back a level of indentation in line 5 of the above indicated
that the \verb|print| statement written there was not part of the inner
loop body (the inner \verb|for| statement had reached its conclusion).

\verb|for| statements are more general than just the ability to iterate
through a sequence of integers:
\begin{verbatim}
>>> print l
[3, 2.0, 'two']
>>> for obj in l:
...   print obj
... 
3
2.0
two
>>> 
\end{verbatim}

More general iterations can be performed with the \verb|while|
statement:
\begin{verbatim}
>>> i = 0
>>> while i < 10:
...   print i
...   i += 1
... 
0
1
2
3
4
5
6
7
8
9
>>> 
\end{verbatim}

Conditional statements are used to choose one branch of code or
another depending on the value of an expression:
\begin{verbatim}
>>> i = 0
>>> while(i < 10):
...   i += 1
...   if i % 2:
...     print "%d is odd" % i
...   else:
...     print "%d is even" % i
... 
1 is odd
2 is even
3 is odd
4 is even
5 is odd
6 is even
7 is odd
8 is even
9 is odd
10 is even
>>> 
\end{verbatim}

\subsection{Functions}

The most basic unit of software in Python is the function. A function
is a portion of code within a larger program which performs a specific
task or computation relatively independently of the rest of the
program.

Functions in Python are defined using the \verb|def| statement and as
seen earlier, indentation determines where they begin and end:
\begin{verbatim}
>>> def square(x):
...   return x*x
... 
>>> def cube(x):
...   return x*square(x)
... 
>>> print square(4)
16
>>> print cube(4)
64
>>>
\end{verbatim}
When a function by its nature has multiple return values, a tuple can
be used to aggregate them:
\begin{verbatim}
>>> def square_and_cube(x):
...   return (square(x), cube(x))
... 
>>> print square_and_cube(4)
(16, 64)
>>> 
\end{verbatim}
Where mutable data-types are involved, beware the potential for
side-effects:
\begin{verbatim}
>>> def foo(l):
...   l[0] = 12
... 
>>> l = [1, 2, 3]
>>> foo(l)
>>> print l
[12, 2, 3]
>>> 
\end{verbatim}
In general, it's probably good advice to avoid programming in this
fashion. Explicit is better than implicit:
\begin{verbatim}
>>> def foo(l):
...   m = l[:] # use slicing to make a (deep) copy of l
...   m[0] = 12
...   return m
... 
>>> l = [1, 2, 3]
>>> m = foo(l)
>>> print l
[1, 2, 3]
>>> print m
[12, 2, 3]
>>> 
\end{verbatim}

\subsection{Classes}

There's not much you can't do with the built-in Python types
(primitives, tuples, lists and dictionaries). Programs though are
typically written once and read again many, many times. The programmer
should strive hard to convey the meaning of the program to the human
reader as much as he/she possibly can. Tuples, lists and dictionaries by
themselves do little to aid semantic comprehension of the data
elements of a program and that realisation has inspired grouping of
related data elements into user-definable aggregate structures from
at least as far back in time as Pascal's user defined \verb|record|
data-type. Python permits the programmer to define his/her own types
together with operations (in Python terminology -- \emph{methods},)
that operate on \emph{instances} of these \emph{class} types.

The most simple \verb|class| in Python can be written like this:
\begin{verbatim}
>>> class empty(object):
...   pass
... 
>>> e=empty()
\end{verbatim}
This isn't a very rich type but it is a type nonetheless. Here's
another user defined \verb|class| that is a bit more interesting:
\begin{verbatim}
>>> class person(object):
...   def __init__(self, age, name):
...     self.age = age
...     self.name = name
...
>>> p = person(25, "John Doe")
>>> print p.age
25
>>> print p.name
John Doe
>>> 
>>> print type(p)
<class '__main__.person'>
\end{verbatim}
The benefits of improved semantic comprehensibility via the
abstraction of the above person \verb|p| over the representation
\verb|(25, "John Doe")| can be augmented even further with class
methods:
\begin{verbatim}
>>> class person(object):
...   def __init__(self, age, name):
...     self.age = age
...     self.name = name
...   def birth_year(self, current_year):
...     return current_year - (self.age + 1)
... 
>>> p = person(25, "John Doe")
>>> print p.birth_year(2008)
1982
\end{verbatim}

Classes can be used to model IS\_A relationships (inheritance):
\begin{verbatim}
>>> class employee(person):
...   def __init__(self, age, name, salary):
...     person.__init__(self, age, name)
...     self.salary = salary
...   def earns(self):
...     return self.salary
... 
>>> e = employee(25, "John Doe", 15000)
>>> print e.birth_year(2008)
1982
>>> print e.earns()
15000
>>> 
\end{verbatim}
Notice how the methods of the \verb|person| class are inherited by the
\verb|employee| class. That is, an \verb|employee| can be substituted
for wherever the context calls for a \verb|person|. Note that any
method in a base class can be overriden in a derived class:
\begin{verbatim}
>>> class manager(employee):
...   def __init__(self, name, age, salary):
...     employee.__init__(self, name, age, salary)
...   def earns(self):
...     raise RuntimeError, "This operation is restricted"
... 
>>> m = manager(25, "John Doe", 20000)
>>> print m.age
25
>>> print m.name
John Doe
>>> print m.birth_year(2008)
1982
>>> print m.earns()
Traceback (most recent call last):
  File "<stdin>", line 1, in <module>
  File "<stdin>", line 5, in earns
RuntimeError: This information is restricted
>>> 
\end{verbatim}
Unlike C++ with its notion of \verb|private|, \verb|protected| and
\verb|public| access mechanisms, everything in a class is publicly
accessible and due to Python's dynamic nature the definition of a
class can even be changed during execution of the script(!):
\begin{verbatim}
>>> def salary(mgr):
...   return mgr.salary
... 
>>> print salary(m)
20000
>>> manager.earns = salary
>>> print m.earns()
20000
>>> 
\end{verbatim}

Programming with classes is ubiquitous in Python and this section just
covers the basics. The newcomer to Python is recommened to take the
time to become more familiar with them early on.

\subsection{Modules and Packages}

A module is a lexical unit of Python code stored in a file on
disk. Let us suppose the existence of a file `complex.py' with
contents something like the following:
\begin{verbatim}
class complex(object):
  def __init__(self, re, im):
    self.re, self.im = (re, im)

  def real_part(self):
    return self.re

  def imag_part(self):
    return self.im

  def __str__(self):
    return "%f + %fi" % (self.re, self.im)

  def __add__(self, other):
    return complex(self.re + other.re, self.im + other.im)

  # ...

def conjugate(x):
  return complex(x.real_part(), -x.imag_part())

# ...
\end{verbatim}
We can use the Python \verb|import| directive to bring in all of the
type declarations and function definitions defined in `complex.py'
into the current scope like this:
\begin{verbatim}
>>> from complex import *
>>> i = complex(0, 1)
>>> print i
0.000000 + 1.000000i
>>> print conjugate(i)
0.000000 + -1.000000i
>>> print i + conjugate(i)
0.000000 + 0.000000i
>>>
\end{verbatim}
We can \verb|import| symbols into the current scope more selectively using
different syntactic forms of the \verb|import| directive:
\begin{verbatim}
>>> from sys import path # sys is a built-in module
>>> import pprint # so is pprint
>>> pprint.pprint(path)
['',
 'C:\\WINDOWS\\system32\\python25.zip',
 'C:\\Python25\\DLLs',
 'C:\\Python25\\lib',
 'C:\\Python25\\lib\\plat-win',
 'C:\\Python25\\lib\\lib-tk',
 'C:\\Python25',
 'C:\\Python25\\lib\\site-packages',
 'C:\\Python25\\lib\\site-packages\\win32',
 'C:\\Python25\\lib\\site-packages\\win32\\lib',
 'C:\\Python25\\lib\\site-packages\\Pythonwin']
>>> \end{verbatim}
Notice that the last form of the \verb|import| directive required that
we prefix our invocation of the \verb|pprint| function with the name
of the module in which it resides (i.e module \verb|pprint|).

A more extensive library for complex numbers would offer different
representations. Imagine now a directory called `complex' with two
files, `cartesian.py' and `polar.py'. The contents of `cartesian.py'
defines the \verb|class complex| as before:
\begin{verbatim}
class complex(object):
  def __init__(self, re, im):
    self.re, self.im = (re, im)

  def real_part(self):
    return self.re

  def imag_part(self):
    return self.im

  def __str__(self):
    return "%f + %fi" % (self.re, self.im)

  def __add__(self, other):
    return complex(self.re + other.re, self.im + other.im)

  # ...

def conjugate(x):
  return complex(x.real_part(), -x.imag_part())

# ...
\end{verbatim}
The contents of `polar.py' defines a different representation for
\verb|class complex|:
\begin{verbatim}
class complex(object):
    def __init__(self, a, theta):
        self.a = a
        self.theta = theta
    def __str__(self):
        return "%fe^i(%f)" % (self.a, self.theta)
    # ...
\end{verbatim}
We create a third file in the `complex' directory, `\_\_init\_\_.py':
\begin{verbatim}
import cartesian
import polar

def polar_to_cartesian(z):
  import math
  x = z.a*math.cos(z.theta)
  y = z.a*math.sin(z.theta)
  return cartesian.complex(x, y)
\end{verbatim}
Complex is now a Python \emph{package}. The following snippet
exercises the package and demonstrates yet another form of the
\verb|import| directive:
\begin{verbatim}
C:\Documents and Settings\PythonUser>python -i
>>> import math # math is a built-in module
>>> import complex
>>> from complex import cartesian as rectangular
>>> from complex import polar as polar
>>> i = rectangular.complex(0, 1)
>>> print i
0.000000 + 1.000000i
>>> i = polar.complex(1, math.pi/2)
>>> print i
1.000000e^i(1.570796)
>>> print complex.polar_to_cartesian(i)
0.000000 + 1.000000i
>>>
\end{verbatim}
Naturally, packages can contain modules and sub-packages which in turn
can contain further modules and sub-packages.

\section{Conclusion}

This has been a whistle-stop tour of the Python language. The basics
have been presented and much detail overlooked. It is very much hoped
that this has been enough to whet the reader's appetite for Python
programming and we strongly encourage the reader to seek out more
detailed references for Python programming.

