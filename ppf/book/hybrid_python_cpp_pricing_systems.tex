\chapter{Hybrid Python/C++ Pricing Systems}\label{ch:hybrid-pricing-systems}

In Chapter \ref{ch:extending-python-from-c++}, we saw how we may
extend Python with modules written in C++. Specifically, we studied
case examples of reflecting types from the Boost.Date\_Time library
into Python. Further, in section \ref{ssec:imm-dates} we developed
functionality in Python making use of those reflected types to compute
IMM dates. In this chapter we aim to show how we can exercise that
functionality from C++. What is the relevance of this? Well, it
demonstrates the possibility of `hybrid pricing systems'. That is,
systems formed from a mix of both C++ and Python. Such potential will
likely be of no small interest to those institutions that already have
a considerable investment in C++. By using the techniques about to be
presented, organisations with pricing frameworks in C++ can enjoy the
many benefits Python offers whilst seamlessly integrating these
efforts with their existing C++ frameworks.
\section{nth\_imm\_of\_year revisited}
Our goal will be to exercise the Python code for IMM date computation
functionality, presented in section \ref{ssec:imm-dates} from C++. As
the first step in seeing how to achieve that, it is necessary for us to
review the implementation of the Python class \verb|nth_imm_of_year|
and examine in closer detail its relationship with the C++ code on
which it rests.
\begin{verbatim}
from ppf_date_time import  \
     weekdays              \
   , months_of_year        \
   , nth_kday_of_month     \
   , year_based_generator

class nth_imm_of_year(year_based_generator):
  '''Calculate the nth IMM date for a given year

  '''
  # ...
\end{verbatim}
We see that this class inherits from the \verb|ppf.date_time| type \\
\verb|class year_based_generator|. The \verb|class year_based_generator|
 is served up to Python in the C++ Python extension module
\verb|ppf_date_time.pyd| and is in fact a
Boost.Date\_Time defined base type for year based generators, or in other 
words, a base class for polymorphic function objects that take a year 
and produce a concrete date. In synopsis:
\begin{verbatim}
template<typename date_type> 
class year_based_generator {
public:
  // types
  typedef date_type::calendar_type calendar_type;
  typedef calendar_type::year_type year_type;    

  // construct/copy/destruct
  year_based_generator();
  ~year_based_generator();

  // public member functions
  virtual date_type get_date(year_type) const;
  virtual std::string to_string() const;
};
\end{verbatim}
In the \verb|ppf_date_time.pyd| source file
`register\_date\_more.cpp', can be found the following code
describing the instantiation of this class template over \\
\verb|class boost::gregorian::date| to Python:
\begin{verbatim}
namespace ppf { namespace date_time {

struct year_based_generator_wrap
  : boost::date_time::
      year_based_generator<boost::gregorian::date>
  , boost::python::wrapper<
     boost::date_time::
      year_based_generator<boost::gregorian::date> >
{
  boost::gregorian::date
    get_date(boost::gregorian::date::year_type y) const
  {
    return this->get_override("get_date")(y);
  }

  std::string to_string() const
  {
    return this->get_override("to_string")();
  }
};

void register_date_more()
{
  using namespace boost::python;
  namespace bg = boost::gregorian;
  namespace bd = boost::date_time;

  // ...

  class_<year_based_generator_wrap
       , boost::noncopyable>("year_based_generator")
    .def("get_date", pure_virtual(&bd::year_based_generator<bg::date>::get_date))
    .def("to_string", pure_virtual(&bd::year_based_generator<bg::date>::to_string))
    ;

  // ...
}

}} // namespace ppf::date_time
\end{verbatim}

\section{Exercising nth\_imm\_of\_year from C++}

We turn our attention now to `embedding' a Python interpreter in a
C++ program. We have seen how to call C++ code from Python. Here we
aim to do the reverse: call Python code from C++. To do this we use a
mix of Python C API routines together with helper types and functions
in the Boost.Python \index{Boost!Python} library:
\begin{verbatim}
#include <boost/detail/lightweight_test.hpp>
#include <boost/python.hpp>
#include <boost/date_time/gregorian/gregorian.hpp>

int main()
{
  namespace bd = boost::date_time;
  namespace bg = boost::gregorian;
  namespace python = boost::python;
  typedef bd::year_based_generator<bg::date> ybd_t;

  Py_Initialize();

  try
  {
    //extract the ppf.date_time.nth_imm_of_year class
    //object
    python::object main_module = python::import("__main__");
    python::object global(main_module.attr("__dict__"));
    python::object result =
      python::exec("from ppf.date_time import *\n", global, global);
    python::object nth_imm_of_year_class = global["nth_imm_of_year"];

    //use the class object to create instances of
    //nth_imm_of_year
    python::object first_imm_ = nth_imm_of_year_class(bg::Mar);
    python::object second_imm_ = nth_imm_of_year_class(bg::Jun);
    python::object third_imm_ = nth_imm_of_year_class(bg::Sep);
    python::object fourth_imm_ = nth_imm_of_year_class(bg::Dec);

    //get references to boost date_time year_based_generators
    //from the newly created objects
    ybd_t& first_imm =  python::extract<ybd_t&>(first_imm_);
    ybd_t& second_imm = python::extract<ybd_t&>(second_imm_);
    ybd_t& third_imm =  python::extract<ybd_t&>(third_imm_);
    ybd_t& fourth_imm = python::extract<ybd_t&>(fourth_imm_);

    //check imm dates for 2005
    BOOST_TEST(first_imm.get_date  (2005) == bg::date(2005, bg::Mar, 16));
    BOOST_TEST(second_imm.get_date (2005) == bg::date(2005, bg::Jun, 15));
    BOOST_TEST(third_imm.get_date  (2005) == bg::date(2005, bg::Sep, 21));
    BOOST_TEST(fourth_imm.get_date (2005) == bg::date(2005, bg::Dec, 21));
  }
  catch(python::error_already_set const&)
  {
    PyErr_Print();
  }
  catch(std::runtime_error const& e)
  {
    std::cerr << e.what() << std::endl;
  }
  catch(...)
  {
    std::cerr << "unexpected exception" << std::endl;
  }

  return boost::report_errors();
}
\end{verbatim}

The significant steps in the above program are as follows: first, the
Python interpreter is initialized with the call to the Python C API
function \verb|Py_Initialize()|. Amongst other things, this has the
effect of creating the fundamental \verb|''__main__''| module. The
following statements
\begin{verbatim}
    python::object main_module = python::import("__main__");
    python::object global(main_module.attr("__dict__"));
\end{verbatim}
import the \_\_main\_\_ module and obtain a reference to its
namespace. With these the context for executing Python code has been
obtained and we execute
\begin{verbatim}
    python::object result =
      python::exec("from ppf.date_time import *\n", global, global);
\end{verbatim}
to cause Python to import all types and functions in the
\verb|ppf.date_time| module into the global namespace . In particular,
we are interested in the \verb|nth_imm_of_year| class object which we
retrieve with
\begin{verbatim}
    python::object nth_imm_of_year_class = global["nth_imm_of_year"];
\end{verbatim} With the class object at our disposal, we use it to
produce \verb|nth_imm_of_year| class instances
\begin{verbatim}
    python::object first_imm_ = nth_imm_of_year_class(bg::Mar);
    python::object second_imm_ = nth_imm_of_year_class(bg::Jun);
    python::object third_imm_ = nth_imm_of_year_class(bg::Sep);
    python::object fourth_imm_ = nth_imm_of_year_class(bg::Dec);
\end{verbatim} This is the point at which things get
interesting. Since \verb|nth_imm_of_year| IS\_A
\verb|year_based_generator<date>| we can manipulate them through a
\verb|year_based_generator<date>| reference or pointer
\begin{verbatim}
    ybd_t& first_imm =  python::extract<ybd_t&>(first_imm_);
    ybd_t& second_imm = python::extract<ybd_t&>(second_imm_);
    ybd_t& third_imm =  python::extract<ybd_t&>(third_imm_);
    ybd_t& fourth_imm = python::extract<ybd_t&>(fourth_imm_);
\end{verbatim} Finally, invoking the virtual function \verb|get_date|
on these references calls out to Python for the implementation
\begin{verbatim}
    BOOST_TEST(first_imm.get_date  (2005) == bg::date(2005, bg::Mar, 16));
    BOOST_TEST(second_imm.get_date (2005) == bg::date(2005, bg::Jun, 15));
    BOOST_TEST(third_imm.get_date  (2005) == bg::date(2005, bg::Sep, 21));
    BOOST_TEST(fourth_imm.get_date (2005) == bg::date(2005, bg::Dec, 21));
\end{verbatim} The net effect is that this shows how we can implement
classes in Python and have them behave as first class C++ types in
C++, a very powerful technique indeed!

