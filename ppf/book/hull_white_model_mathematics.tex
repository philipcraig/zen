\chapter{Hull-White model mathematics} \label{appendix:hw}

In this appendix we give a brief outline of the Hull-White model. For
a more in-depth discussion of the Hull-White model, readers are
encouraged to consult \cite{book:JCHULL} and \cite{book:REBONATO}. The
Hull-White model belongs to a class of HJM models called extended
Vasicek models. This class of one-factor models has, in the risk
neutral measure denoted by $\mathbb Q$, the following short rate
process
\begin{equation}
dr(t) = \left(m(t)-\lambda(t)r(t)\right) dt+\sigma(t) d W(t)
\end{equation}
with $r(t)$, the short rate at time $t$, $m(t),\lambda(t),\sigma(t):
\mathbb R^+ \mapsto \mathbb R^+$ and $W(t)$ is a $\mathbb Q$-brownian
motion. The original Hull-White model makes the further simplification
that both $\sigma$ and $\lambda$ are constant in time. Introducing the
auxiliary variables defined below
\begin{eqnarray}
C(t) &:=& \sigma(t) \exp(\lambda t) \\
\phi(t) &:=& \frac{1-\exp(-\lambda t)}{\lambda} \\
M(t) &:=& \exp(-\lambda t) r(0) + \int_0^t \exp(\lambda(s-t)) m(s) ds
\end{eqnarray}
it is straightforward to show
\begin{equation}
R(t,T) := \int_t^T r(s)ds = \int_t^T  M(s) ds + \int_t^T \left(\phi(T)-\phi(s)\right) C(s) dW(s).
\end{equation}

The stochastic discount factor and zero coupon bond can be expressed
in terms of $R(t,T)$ as follows:
\begin{eqnarray}
B(t)^{-1} &:=&  \exp(-R(0,t)) \\
P(t,T) &:=& \mathbb E \left[ \exp(-R(t, T) )| \mathcal F_t \right]
\end{eqnarray}
where $\mathbb E$ denotes the expectation in the risk neutral measure
and $\mathcal F_t$ the filtration at time $t$. Before carrying out the
above expectations we note that $B(t)$ is called the money-market
account and is the numeriare in the risk-neutral measure. Performing
the expectations we obtain
\begin{eqnarray} \label{eq:discount}
B(t)^{-1} &=& P(0, t) \exp( -\int_0^t \left(\phi(t)-\phi(s)\right) C(s) dW(s) \\
&&- \frac{1}{2} \int_0^t \left(\phi(t)-\phi(s)\right)^2 C(s)^2 ds)
\end{eqnarray}
and
\begin{eqnarray} \label{eq:zcb}
P(t, T) &=& \frac{P(0,T)}{P(0,t)} \exp(-\left(\phi(T)-\phi(t)\right) \int_0^t C(s) dW(s)  \nonumber \\
&& -\frac{1}{2} \int_0^t \left(\phi(T)-\phi(t)\right)\left(\phi(T)+\phi(t)-2\phi(s)\right) C(s)^2 ds).
\end{eqnarray}
Note that as expected the stochastic discount factor is a $\mathbb
Q$-martingale, in fact it is an exponential martingale, whereas the
zero coupon bond price is not a $\mathbb Q$-martingale because, as can
be seen below, its SDE has a non-zero drift.
\begin{equation} dP(t,T) = P(t,T) \left(r(t) dt +
\left(\phi(t)-\phi(T)\right) C(t) dW(t)\right).
\end{equation}

For non path-dependent pricing problems it is normally convenient to
work in the so called forward $\mathbb Q^T$-measure. In this measure
the numeraire at time $t$ is simply $P(t,T)$ and Girsanov's
theorem implies that $\bar{W}(t)$, as defined below, is a $\mathbb
Q^T$-Brownian motion

\begin{equation} \label{eq:com}
d\bar{W}(t) = dW(t) + \left(\phi(T)-\phi(t)\right)C(t) dt.
\end{equation}

Substitution of equation (\ref{eq:com}) into equation (\ref{eq:zcb}) yields
\begin{eqnarray} \frac{P(t,T^{'})}{P(t,T)} &=&
\frac{P(0,T^{'})}{P(0,T)} \exp(-\left(\phi(T^{'})-\phi(T)\right)
\int_0^t C(s) d\bar{W}(s) \nonumber \\ &&-\frac{1}{2}
\left(\phi(T^{'})-\phi(T)\right)^2 \int_0^t C(s)^2 ds), ~~~\forall t
\le T^{'} \le T.
\end{eqnarray}

In other words the numeriare rebased zero coupon bond in the forward
$\mathbb Q^T$-measure is a $\mathbb Q^T$-martingale. This to be
expected in complete markets, where all numeriare rebased tradeables
are martingales. Indeed, let $V(t)$ denote the value at $t$ of any
tradeable, then by the martingale property we have
\begin{equation} V(s) = P(s,T) \mathbb E^{\mathbb
Q^T}[\frac{V(t)}{P(t,T)} | \mathcal F_s], ~~~\forall s \le t.
\end{equation}
For an excellent introduction to financial calculus covering
everything from measures, filtrations to martingales and arbitrage
free pricing please consult \cite{BOOK:RENNIE}.

