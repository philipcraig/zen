\chapter{Extending Python from C++}\label{ch:extending-python-from-c++}

It is usual in financial institutions that make use of quantitative
analysis programs to have a considerable investment in C++.  Thus it
can be important then to foster interoperability between C++ and
Python. This chapter studies how Python modules can be implemented in
C++ by means of the Boost.Python\footnote{Boost provides free
peer-reviewed portable C++ source libraries. See http://www.boost.org
for details.}  library (see also appendix \ref{appendix:boost-python}
for a primer on the Boost.Python library).

\section{Boost.Date\_Time types} \label{sec:boost-date-time}

It is common in quantitative analysis programming to require
manipulation of and computations involving dates.  The `Python
Library' contains excellent functionality for such activities.
Pricing systems written in C++ however will be implemented using C++
datatypes for the representation of dates and times. For pricing
frameworks implemented in a hybrid of Python and C++, it would be
convenient to settle on a common representation of these fundamental
types. Accordingly, in this section we demonstrate the `reflection' of
functionality from the C++ Boost.Date\_Time library to Python.

Our reflection of the C++ date types into Python will be housed in the
Python module `ppf\_date\_time.pyd', implemented in C++.  We declare
this intention in the entry point to our Python module in the file
`module.cpp':
\begin{verbatim}
#include <boost/python/module.hpp>

namespace ppf
{
  namespace date_time
  {
    void register_date();
    void register_date_more();
    
  } // namespace date_time

} // namespace ppf

BOOST_PYTHON_MODULE(ppf_date_time)
{
  using namespace ppf::date_time;

  register_date();
  register_date_more();
}

\end{verbatim}
In `register\_date.cpp' we instantiate Boost.Python
\verb|class_| objects describing the C++ types and functions we intend
to use from Python:
\begin{verbatim}
void register_date()
{
  using namespace boost::python;
  namespace bg = boost::gregorian;
  namespace bd = boost::date_time;

  // types and functions ...

  class_<bg::date>(
      "date"
     ,"A date type based on the gregorian calendar"
     , init<>("Default construct not_a_date_time"))
    .def(init<bg::date const&>())
    .def(init
         <
           bg::greg_year
         , bg::greg_month
         , bg::greg_day
         >((arg("y"), arg("m"), arg("d"))
       , "Main constructor with year, month, day "))
    .def("year", &bg::date::year)
    .def("month", &bg::date::month)
    .def("day", &bg::date::day)

    // ...

    ;

  class_<std::vector<bg::date> >(
      "date_vec"
    , "vector (C++ std::vector<date> ) of date")
    .def(vector_indexing_suite<std::vector<bg::date> >())
    ;

  // more types and functions ...
}
\end{verbatim}
Once exposed in this fashion, the types so defined in the
\verb|ppf_date_time| module are imported into the \verb|ppf| subpackage
\verb|ppf.date_time| by means of import statements in the module's
`\_\_init\_\_.py':
\begin{verbatim}
from ppf_date_time import *
\end{verbatim}

\subsection{Examples}
\subsubsection{IMM dates}\label{ssec:imm-dates}
As an example of what we have achieved, let's see how in Python, we
can compute so called IMM (international money market) dates for a
given year i.e. the 3rd Wednesday of March, June, September, and
December in the year. The \verb|ppf.date_time| package provides the
module \verb|nth_imm_of_year| in which is defined \verb|class nth_imm_of_year|. 
The workhorse of the class implementation is the
Boost.Date\_Time function \verb|nth_kday_of_month|:
\begin{verbatim}
from ppf_date_time import  \
     weekdays              \
   , months_of_year        \
   , nth_kday_of_month     \
   , year_based_generator

class nth_imm_of_year(year_based_generator):
  '''Calculate the nth IMM date for a given year

  '''
  first = months_of_year.Mar
  second = months_of_year.Jun
  third = months_of_year.Sep
  fourth = months_of_year.Dec

  def __init__(self, which):
    year_based_generator.__init__(self)
    self._month = which

  def get_date(self, year):
    return nth_kday_of_month(
          nth_kday_of_month.third
        , weekdays.Wednesday
        , self._month).get_date(year)

  def to_string(self):
    pass
\end{verbatim}
Exercising the \verb|class nth_imm_of_year| functionality in an
interactive Python session goes like this:
\begin{verbatim}
>>> from ppf.date_time import *
>>> imm = nth_imm_of_year
>>> imm_dates = []
>>> imm_dates.append(imm(imm.first).get_date(2005))
>>> imm_dates.append(imm(imm.second).get_date(2005))
>>> imm_dates.append(imm(imm.third).get_date(2005))
>>> imm_dates.append(imm(imm.fourth).get_date(2005))
>>> for t in imm_dates:
...   print t
2005-Mar-16
2005-Jun-15
2005-Sep-21
2005-Dec-21
\end{verbatim}

With \verb|class nth_imm_of_year| some useful questions regarding IMM
dates can now be answered elegantly and easily. For example, what is
the IMM date immediately preceding a given date? This is answered in
the \\ \verb|ppf.date_time.first_imm_before| module:
\begin{verbatim}
from ppf_date_time import  \
     weekdays              \
   , months_of_year        \
   , nth_kday_of_month     \
   , year_based_generator
from nth_imm_of_year import *

def first_imm_before(start):
  '''Find the IMM date immediately preceding the given date.
  '''
  imm = nth_imm_of_year
  first_imm_of_year = imm(imm.first).get_date(start.year())
  imm_date = None
  if start <= first_imm_of_year:
    imm_date = imm(imm.fourth).get_date(start.year() - 1)
  else:
    for imm_no in reversed([imm.first, imm.second, imm.third, imm.fourth]):
      imm_date = imm(imm_no).get_date(start.year())
      if imm_date < start:
        break

  return imm_date
\end{verbatim}
In an interactive Python session:
\begin{verbatim}
>>> from ppf.date_time import *
>>> print first_imm_before(date(2007, Jun, 27))
2007-Jun-20
\end{verbatim}
The \verb|ppf.date_time| package also contains the symmetric \verb|first_imm_after| function.
\subsubsection{Holidays, Rolls and Year Fractions}
Other common activities in financial modelling include determining if a
date is a business day, `rolling' a date to a business day and the
computation of elapsed time between two dates according to common
market conventions.

The \verb|ppf.date_time.shift_convention| module shows an easy way to
emulate C++ enum types:
\begin{verbatim}
class shift_convention:
    none                 \
  , following            \
  , modified_following   \
  , preceding            \
  , modified_preceding = range(5)
\end{verbatim}
This idiom is employed again in the
\verb|ppf.date_time.day_count_basis| module:
\begin{verbatim}
class day_count_basis:
    basis_30360   \
  , basis_act_360 \
  , basis_act_365 \
  , basis_act_act = range(4)
\end{verbatim}
The \verb|ppf.date_time.is_business_day| module provides the means to
answer the question of whether or not a given date is a business day
\begin{verbatim}
from ppf_date_time import weekdays

def is_business_day(t, financial_centers=None):
  ''' Test whether the given date is a business day.
      In this version, only weekends are considered
      holidays.
  '''
  Saturday, Sunday = weekdays.Saturday, weekdays.Sunday

  return t.day_of_week().as_number() != Saturday \
       and t.day_of_week().as_number() != Sunday

\end{verbatim}
The \verb|ppf.date_time.shift| module provides functionality to
`shift' a date according to the common market shift conventions:
\begin{verbatim}
from ppf_date_time import *
from is_business_day import *
from shift_convention import *

def shift(t, method, holiday_centers=None):
  d = date(t)
  if not is_business_day(d):
    if method == shift_convention.following:
      while not is_business_day(d, holiday_centers):
        d = d + days(1)
    elif method == shift_convention.modified_following:
      while not is_business_day(d, holiday_centers):
        d = d + days(1)
      if d.month().as_number() != t.month().as_number():
          d = date(t)
          while not is_business_day(d, holiday_centers):
            d = d - days(1)
    elif method == shift_convention.preceding:
      while not is_business_day(d, holiday_centers):
        d = d - days(1)
    elif method == shift_convention.modified_preceding:
      while not is_business_day(d, holiday_centers):
        d = d - days(1)
      if d.month().as_number() != t.month().as_number():
        while not is_business_day(d, holiday_centers):
          d = d + days(1)
    else: raise RuntimeError, "Unsupported method"

  return d
\end{verbatim}
The \verb|ppf.date_time.year_fraction| module provides functionality
to compute year fractions:
\begin{verbatim}
from ppf_date_time \
     import date, gregorian_calendar_base
from day_count_basis import *

is_leap_year = gregorian_calendar_base.is_leap_year

def year_fraction(start, until, basis):
  '''Compute accruals
  '''
  result = 0
  if basis == day_count_basis.basis_act_360:
    result = (until - start).days()/360.0
  elif basis == day_count_basis.basis_act_365:
    result = (until - start).days()/365.0
  elif basis == day_count_basis.basis_act_act:
    if start.year() != until.year():
      start_of_to_year = date(until.year(), 1, 1)
      end_of_start_year = date(start.year(), 12, 31)
      result = (end_of_start_year - start).days()/ \
          (365.0, 366.0)[is_leap_year(start.year())] \
        +  (int(until.year()) - int(start.year()) - 1) + \
           (until - start_of_to_year).days()/ \
              (365.0, 366.0)[is_leap_year(until.year())]
    else:
      result = (until - start).days()/ \
               (365.0, 366.0)[is_leap_year(util.year())]
  elif basis == day_count_basis.basis_30360:
    d1, d2 = start.day(), until.day()
    if d1 == 31:
        d1 -= 1
    if d2 == 31:
        d2 -= 1
    result = (int(d2) - int(d1)) + \
             30.0*(int(until.month()) - int(start.month())) + \
                      360.0*(int(until.year()) - int(start.year()))
    result = result / 360.0
  else:
    raise RuntimeError, "Unsupported basis"

  return result
\end{verbatim}
In the following interactive session, the year fraction between
two dates is computed under a variety of different day count basis
conventions:
\begin{verbatim}
>>> from ppf.date_time import *
>>> add_months = month_functor
>>> Nov = months_of_year.Nov
>>> begin = date(2004, Nov, 21)
>>> until = begin + add_months(6).get_offset(begin)
>>> year_fraction(begin, until, day_count_basis.basis_30360)
0.5
>>> year_fraction(begin, until, day_count_basis.basis_act_365)
0.49589041095890413
>>> year_fraction(begin, until, day_count_basis.basis_act_act)
0.49285126132195523
\end{verbatim}

\section{Boost.MultiArray and Special Functions}

The use of multi-dimensional arrays in quantitative analysis programs
is ubiquitous. Python, or rather the Python libraries provide a
variety of types that serve for their representation. Like the date
types of the previous section however, we prefer to emphasize
interoperability with C++ and so, to this end, might favour reflection
of C++ array types into Python. The \verb|ppf| package exposes the
Boost.MultiArray multi-dimensional array types \\
\verb|boost::multi_array<double,N>| for \verb|N|$ = 1, 2, 3$ to
Python. To achieve this, advantage was taken of a C++ template
meta-program that facilitates reflection of the arrays, the code for
which is present in the source code accompanying this book (see
`ext/boost/multi\_array/multi\_array.hpp').

The array types are housed in the \verb|ppf_math| module implemented
in the C++ Python extension `ppf\_math.pyd' and imported into
the namespace of the \verb|ppf.math| sub-package. Usage of the array
types is natural and intuitive. Here is an example taken from the
\verb|ppf.math| unit-tests:
\begin{verbatim}
class solve_upper_diagonal_system_tests(unittest.TestCase):
  def test(self):
    
  # Solve upper diagonal system of linear equations ax = b
  # where
  #
  # a = 3x3
  #     [  1.75    1.5    -2.5
  #         0     -0.5    0.65
  #         0        0    0.25 ]
  #
  # and b = [0.5, -1.0, 3.5].
  
  a = ppf.math.array2d([3,3])
  a[0, 0], a[0, 1], a[0, 2] = (1.75, 1.5, -2.5)
  a[1, 0], a[1, 1], a[1, 2] = (0.0, -0.5,  0.65)
  a[2, 0], a[2, 1], a[2, 2] = (0.0,  0.0,  0.25)

  b = ppf.math.array1d([3])
  b[0] =  0.5
  b[1] = -1.0
  b[2] =  3.5
  
  # Expected solution vector is x = [2.97142857  20.2  14.0].
  
  x = ppf.math.solve_upper_diagonal_system(a, b)
  assert len(x) == 3 and math.fabs(x[0] - 2.971428571) < 1.0e-6 \
         and math.fabs(x[1] - 20.2) < 1.0e-6 and math.fabs(x[2] - 14.0) < 1.0e-6
\end{verbatim}

In addition to the multi-array types, the module \verb|ppf_math|
also exposes some useful utility functions implemented in C++. In
the file `ppf/math/limits.hpp' are the following template
function definitions:
\begin{verbatim}
#if !defined(LIMITS_5DDE828B_9989_44F5_9728_47AA72323D96_INCLUDED)
#  define LIMITS_5DDE828B_9989_44F5_9728_47AA72323D96_INCLUDED

#  if defined(_MSC_VER) && (_MSC_VER >= 1020)
#    pragma once
#  endif // defined(_MSC_VER) && (_MSC_VER >= 1020)

#  include <boost/config.hpp>

#  include <limits>

namespace ppf { namespace math {

template <class T>
T epsilon()
{
  return std::numeric_limits<T>::epsilon();
}

template <class T>
T min BOOST_PREVENT_MACRO_SUBSTITUTION ()
{
  return (std::numeric_limits<T>::min)();
}

template <class T>
T max BOOST_PREVENT_MACRO_SUBSTITUTION ()
{
  return (std::numeric_limits<T>::max)();
}

}} // namespace ppf::math

#endif // !defined(LIMITS_5DDE828B_9989_44F5_9728_47AA72323D96_INCLUDED)
\end{verbatim}
In `ext/lib/math/src/register\_special\_functions.cpp', instantiations
of these templates are exposed to Python:
\begin{verbatim}
#include <boost/python/def.hpp>

#include <ppf/math/limits.hpp>

namespace ppf { namespace math {

void register_special_functions()
{
  using namespace boost::python;

  def("epsilon", epsilon<double>);
  def("min_flt", min BOOST_PREVENT_MACRO_SUBSTITUTION <double>);
  def("max_flt", max BOOST_PREVENT_MACRO_SUBSTITUTION <double>);
}

}} // namespace ppf::math
\end{verbatim}
An example of the use of the \verb|epsilon| function is again provided by
a \verb|ppf.math| unit-test:
\begin{verbatim}
class bisect_tests(unittest.TestCase):
  def test1(self):
    tol = 5*ppf.math.epsilon()
    left, right, num_its = \
          ppf.math.bisect(lambda x: x*x + 2.0*x - 1.0
                         , -3, -2
                         , lambda x, y: math.fabs(x-y) < tol, 100)
\end{verbatim}
Further examples of the use of these special functions can be found in
the next chapter.

\section{NumPy arrays}

Despite the efforts of the preceding section regarding reflection of \\
C++ Boost.MultiArray types into Python, in practice, when working in
Python, the authors have found the facilities of NumPy arrays to be
far more convenient (NumPy was mentioned briefly in section
\ref{sec:common-misconceptions-about-python}). Specifically, their
notational conveniences and the large body of functionality provided
by the NumPy library motivates their use in Python beyond the argument
of C++ interoperability. Indeed, when working in C++, a library
dedicated to scientific manipulation of arrays such as
Blitz++\footnote{Blitz++ is a C++ class library for scientific
computing which provides performance on par with Fortran 77/90. See
http://www.oonumerics.org/blitz for details.} wins the authors' favour
for such work over `lower-level' container types like native C arrays
or Boost.MultiArray types. But now the crux of the matter. If we
haven't made this point earlier then we'll make it for the first time
now. One of the great strengths of Python is the ability to drop into
C or C++ code `when performance counts'. That is, the ability to
factor out that characteristic operation that must be done as
efficiently as possible and pull it down into a compiled component is
key. Now, in the field of numerical programming doesn't that characteristic
operation almost always involve operating on arrays of data?

So, can we have it all? Can we have the convenience of NumPy in Python
combined with the convenience and efficiency of Blitz++ in C++ where
the data is shared between these array types? The short answer is yes
we can, as we will demonstrate in next subsection.

\subsection{Accessing array data in C++}

This subsection is concerned with the topic of accessing a NumPy
array's data in C++. To do this, we need to work with the Python C API
and we'll also take advantage of Boost.Python where we can. The
approach is fairly idiomatic and can be more or less wrapped up in a
set of reasonably small utility functions. Let's begin with this most
simple of functions from `ppf/util/python/detail/decref.hpp':
\begin{verbatim}
#if !defined(DECREF_4A1F1D9D_CE18_4CA1_AF52_DA1C51847FB4_INCLUDED)
#  define DECREF_4A1F1D9D_CE18_4CA1_AF52_DA1C51847FB4_INCLUDED

#  if defined(_MSC_VER) && (_MSC_VER >= 1020)
#    pragma once
#  endif // defined(_MSC_VER) && (_MSC_VER >= 1020)

#  include <boost/python/detail/wrap_python.hpp>

namespace ppf { namespace util { namespace python {

namespace detail
{
  //Py_DECREF() is a macro which makes it unsuitable
  //for use with bind constructs in scope guards.
  template <class T>
  inline void decref(T* obj)
  {
    Py_DECREF(obj);
  }
}

}}} // namespace ppf::util::python

#endif // !defined(DECREF_4A1F1D9D_CE18_4CA1_AF52_DA1C51847FB4_INCLUDED)
\end{verbatim}
The motivation for this function will become apparent in a moment, but
briefly, Python objects in the Python C-API are reference counted and
the manipulation of the reference counts (although automatic in
Python) must be carried out manually in C++. As the comment in the
code above indicates, the facilitity for decrementing the reference
count of a Python object is actually a macro and so we need a wrapper
for it should we wish to take advantage of `scope guard'\footnote{See
``Generic: Change the way you write exception safe code - forever'' by
Andrei Alexandrescu and Petru Marginean, available online at
http://www.ddj.com/cpp/184403758} techniques.

Here is the code from `ppf/util/python/detail/object\_as\_array.hpp'
that wraps up the business of getting us from a Python C API
\verb|PyObject*| to a NumPy \verb|PyArrayObject*|:
\begin{verbatim}
#if !defined(OBJECT_AS_ARRAY_0067910E_F5F1_4BD6_9565_3BF98B4A12C1_INCLUDED)
#  define OBJECT_AS_ARRAY_0067910E_F5F1_4BD6_9565_3BF98B4A12C1_INCLUDED

#  if defined(_MSC_VER) && (_MSC_VER >= 1020)
#    pragma once
#  endif // defined(_MSC_VER) && (_MSC_VER >= 1020)

#include <ppf/util/python/detail/decref.hpp>

#include <boost/python/errors.hpp>
#include <boost/shared_ptr.hpp>
#include <boost/bind.hpp>

namespace ppf { namespace util { namespace python {

namespace detail
{

template <int = 0>
struct object_as_array_impl_
{
  static boost::shared_ptr<PyArrayObject> get(
    PyObject* input, int type, int min_dim, int max_dim)
  {
    if(PyArray_Check(input))
    {
      if(!PyArray_ISCARRAY(reinterpret_cast<PyArrayObject*>(input)))
      {
        PyErr_SetString(PyExc_TypeError, "not a C array");
  
        boost::python::throw_error_already_set();
      }
  
      return boost::shared_ptr<PyArrayObject>(
               reinterpret_cast<PyArrayObject*>(
                        boost::python::expect_non_null(
                             PyArray_ContiguousFromObject(
                                     input, type, min_dim, max_dim))
              )
            , boost::bind(
               ::ppf::util::python::detail::decref<PyArrayObject>, _1)
        );
    }
    else
    {
      PyErr_SetString(PyExc_TypeError, "not an array");
  
      boost::python::throw_error_already_set();
    }
  
    return boost::shared_ptr<PyArrayObject>();
  }
};

typedef object_as_array_impl_<> object_as_array_impl;

inline boost::shared_ptr<PyArrayObject>
object_as_array(
  PyObject* input, int type, int min_dim, int max_dim)
{
  return object_as_array_impl::get(input, type, min_dim, max_dim);
}

}}}} // namespace ppf::util::python::detail

#endif // !defined(OBJECT_AS_ARRAY_0067910E_F5F1_4BD6_9565_3BF98B4A12C1_INCLUDED)
\end{verbatim}
Note that this code lives in the \verb|ppf::util::python::detail|
namespace and is not tied to any particular \verb|ppf| C++ Python
extension module.

To explain this code, let's work top down rather than bottom up and
look to the last function of the file first.
\begin{verbatim}
inline boost::shared_ptr<PyArrayObject>
object_as_array(
  PyObject* input, int type, int min_dim, int max_dim)
{
  return object_as_array_impl::get(input, type, min_dim, max_dim);
}
\end{verbatim}
The first thing to note is the return type, that is a \\
\verb|boost::shared_ptr<PyArrayObject>|. The reason to return one of
these over a raw \verb|PyArrayObject*| is to do with the use of
\verb|Py_DECREF| as alluded to above. The long and the short of it is
that should the attempt to get an array from a \verb|PyObject*|
succeed, by the time the resultant array is going out of scope it must
have \verb|Py_DECREF| called on it to avoid a resource leak. This
should happen even in the event of a C++ exception. As we will see,
the wrapping of the array up in the \verb|shared_ptr| means this will
be automated for us.

A quick explanation of the arguments: \verb|input| is the incoming
\verb|PyObject*| which we hope is an array; the \verb|type| is the
expected element type, for \verb|ppf| purposes this is always the
constant \verb|PyArray_DOUBLE|; the arguments \verb|min_dim| and
\verb|max_dim| are the expected minimum dimension (guarantee no smaller than) and maximum
dimension of the array (guarantee no larger than -- if \verb|max_dim|
is set to zero, the check on the array will have no upper bound with
respect to dimensions).

The body of this inline function delegates to a static function
\verb|get| of class type \verb|object_as_array_impl|. The type
\verb|object_as_array_impl| is
in fact a typedef for a specific instantiation of a template class\\
\verb|template <int> class object_as_array_impl_|. That is nothing to
really stop and concern ourselves too much with; it's a fairly often 
observed C++ `trick' that enables us to present this functionality from
a C++ header file without the need to provide clients of the
functionality compiled library code as well.

So, with the interface function covered, a quick review of the
implementation details of the \verb|get| function.
\begin{verbatim}
  static boost::shared_ptr<PyArrayObject> get(
    PyObject* input, int type, int min_dim, int max_dim)
  {
    if(PyArray_Check(input))
    {
      if(!PyArray_ISCARRAY(reinterpret_cast<PyArrayObject*>(input)))
      {
        PyErr_SetString(PyExc_TypeError, "not a C array");
  
        boost::python::throw_error_already_set();
      }
  
      return boost::shared_ptr<PyArrayObject>(
               reinterpret_cast<PyArrayObject*>(
                        boost::python::expect_non_null(
                             PyArray_ContiguousFromObject(
                                     input, type, min_dim, max_dim))
              )
            , boost::bind(
               ::ppf::util::python::detail::decref<PyArrayObject>, _1)
        );
    }
    else
    {
      PyErr_SetString(PyExc_TypeError, "not an array");
  
      boost::python::throw_error_already_set();
    }
  
    return boost::shared_ptr<PyArrayObject>();
  }

\end{verbatim}
Well, it's fairly easy to see that bar a few wrinkles that we'll
discuss in a moment, it's for the most part fairly standard Python C
API style programming. In short, the incoming \verb|input| is checked
to ensure it's an array and that if it is that it be a standard C
style array (row major). In the event that it fails to meet these
conditions, the error is indicated to Python and a quick exit is made
by calling the Boost.Python function
\verb|throw_error_already_set()|. If the object has been determined to
be an array, the crucical call is made to the NumPy API function
\verb|PyArray_ContiguousFromObject| which for our intent checks that
the array has the requested element data type and dimensionality. The
use of the Boost.Python \verb|expect_non_null| is the means by which we
detect if those conditions have been met (the result of
\verb|PyArray_ContiguousFromObject| will be non-null or 0 if they have
not); the Python error indicator is set and an implicit call to
\verb|boost::python::throw_error_already_set()| will occur on failure.
The non-null array object is cast to the required type and
installed into a boost shared pointer with a custom deleter built from
a \verb|boost::bind| to the \verb|ppf::util::python::detail::decref|
function.

\subsection{Examples}

The `ppf\_math.pyd' C++ Python extension module exports some
(trivial) examples of manipulating NumPy arrays from C++. The code for
these examples can be found in the source file
`lib/math/src/register\_numpy.cpp'. The examples all live in the C++
namespace \verb|ppf::math::numpy::examples|. They are available
through the \verb|ppf.math.numpy_examples| module by the names
\verb|sum_array|, \verb|trace|, \verb|assign_zero| and
\verb|make_array|.  The code to register the functions in the
`ppf\_math.py' module reads
\begin{verbatim}
void register_numpy()
{
  using namespace boost::python;

  def("numpy_sum_array", numpy::examples::sum_array);
  def("numpy_trace", numpy::examples::trace);
  def("numpy_assign_zero", numpy::examples::assign_zero);
  def("numpy_make_array", numpy::examples::make_array);

  import_array();//this is a required NumPy API function call
}
\end{verbatim}

\subsubsection{Sum the elements of an array}

The first example simply sums the elements of the incoming array.
\begin{verbatim}
double sum_array(PyObject* input)
{
  boost::shared_ptr<PyArrayObject> obj = ::ppf::util::python
        ::detail::object_as_array(input, PyArray_DOUBLE, 0, 0);

  // compute size of array 
  int n = 1;
  if(obj->nd > 0)
    for(int i = 0; i < obj->nd; ++i) 
      n *= obj->dimensions[i];

  double* array = reinterpret_cast<double*>(obj->data);

  return std::accumulate(array, array + n, 0.);
}
\end{verbatim}
In Python:
\begin{verbatim}
>>> import numpy
>>> from ppf.math.numpy_examples import *
>>> a = numpy.array([1., 2., 3., 4.])
>>> print sum_array(a)
10.0
\end{verbatim}

\subsubsection{Compute the trace of an array}

The next function computes the trace of a two dimensional array
(sum of the main diagonal elements).
\begin{verbatim}
double trace(PyObject* input)
{
  boost::shared_ptr<PyArrayObject> obj = ::ppf::util::python
         ::detail::object_as_array(input, PyArray_DOUBLE, 2, 2);

  int n = obj->dimensions[0];
  if(n > obj->dimensions[1]) n = obj->dimensions[1];

  double sum = 0.;
  for(int i = 0; i < n; ++i)
    sum += *reinterpret_cast<double*>(
      obj->data + i*obj->strides[0] + i*obj->strides[1]);

  return sum;
}
\end{verbatim}
Continuing the above example interpreter session:
\begin{verbatim}
>>> a = numpy.zeros((3, 4))
>>> for i in range(3):
...   a[i, i] = 1
... 
>>> a[2, 3] = 1
>>> print a
[[ 1.  0.  0.  0.]
 [ 0.  1.  0.  0.]
 [ 0.  0.  1.  1.]]
>>> print trace(a)
3.0
\end{verbatim}

\subsubsection{Assign an array's contents to zero}

This function does more than just compute something from an array
defined in Python. It shares the underlying data with a Blitz++ array
in C++ and affects the source array by setting all of its elements to
zero.
\begin{verbatim}
void assign_zero(PyObject* input)
{
  boost::shared_ptr<PyArrayObject> obj = ::ppf::util::python
         ::detail::object_as_array(input, PyArray_DOUBLE, 2, 2);

  blitz::Array<double, 2> array(
      reinterpret_cast<double*>(obj->data)
    , blitz::shape(obj->dimensions[0], obj->dimensions[1])
    , blitz::neverDeleteData);

  array = 0;
}
\end{verbatim}
Continuing on in the interpreter:
\begin{verbatim}
>>> print a
[[ 1.  0.  0.  0.]
 [ 0.  1.  0.  0.]
 [ 0.  0.  1.  1.]]
>>> assign_zero(a)
>>> print a
[[ 0.  0.  0.  0.]
 [ 0.  0.  0.  0.]
 [ 0.  0.  0.  0.]]
\end{verbatim}

\subsubsection{Create a new NumPy array in C++ and return it to Python}

This code creates a new one dimensional array of extent \verb|n| where
\verb|n| is provided by the caller and assigns it the values $0, ..., n - 1$.
\begin{verbatim}
PyObject* make_array(int n)
{
  int dimensions[1]; dimensions[0] = n;
  PyArrayObject* result =
    reinterpret_cast<PyArrayObject*>(
        boost::python::expect_non_null(
          PyArray_FromDims(1, dimensions, PyArray_DOUBLE)));
  double* buffer = reinterpret_cast<double*>(result->data);
  for(int i = 1; i < n; ++i) buffer[i] = i;

  return PyArray_Return(result);
}
\end{verbatim}
Back to the interpreter for one last time:
\begin{verbatim}
>>> a = make_array(6)
>>> print a
[ 0.  1.  2.  3.  4.  5.]

\end{verbatim}
